\documentclass[a4paper,UTF8]{article}
\usepackage{ctex}
\usepackage[margin=1.25in]{geometry}
\usepackage{color}
\usepackage{graphicx}
\usepackage{amssymb}
\usepackage{amsmath}
\usepackage{amsthm}
\usepackage{bm}
\usepackage{hyperref}
\numberwithin{equation}{section}
%\usepackage[thmmarks, amsmath, thref]{ntheorem}
\theoremstyle{definition}
\newtheorem*{solution}{Solution}
\newtheorem*{prove}{Proof}
\usepackage{multirow}

%--

%--
\begin{document}
\title{习题二}
\author{141220120, 徐世坚, xsj13260906215@gmail.com}
\maketitle
\section{[10pts] Lagrange Multiplier Methods}
请通过拉格朗日乘子法(可参见教材附录B.1)证明《机器学习》教材中式(3.36)与式(3.37)等价。即下面公式\eqref{primal}与\eqref{dual}等价。
\begin{equation}
\label{primal}
\begin{split}
 \min_{\mathbf{w}} \quad &-\mathbf{w}^\mathrm{T} \mathbf{S}_b\mathbf{w}\\ 
\text{s.t.} \quad &\mathbf{w}^\mathrm{T} \mathbf{S}_w\mathbf{w} = 1
\end{split}
\end{equation}

\begin{equation}
\label{dual}
\mathbf{S}_b\mathbf{w} = \lambda \mathbf{S}_w\mathbf{w}
\end{equation}
\begin{prove}
此处用于写证明(中英文均可)\\
建立拉格朗日函数如下:\\
$F(w,\lambda)=-w^T S_b w+\lambda (w^T S_w w-1)$\\
$\because S_b$ 和 $ S_w $都是对称阵\\
$\therefore F^{'}_w = -2 S_b w + 2 \lambda S_w w = 0 $\\
$\therefore S_b w = \lambda S_w w$

\qed
\end{prove}

\section{[20pts] Multi-Class Logistic Regression}
教材的章节3.3介绍了对数几率回归解决二分类问题的具体做法。假定现在的任务不再是二分类问题,而是多分类问题,其中$y\in\{1,2\dots,K\}$。请将对数几率回归算法拓展到该多分类问题。

(1) \textbf{[10pts]} 给出该对率回归模型的“对数似然”(log-likelihood);

(2) \textbf{[10pts]} 计算出该“对数似然”的梯度。

提示1:假设该多分类问题满足如下$K-1$个对数几率,
\begin{eqnarray*}
\ln\frac{p(y=1|\mathbf{x})}{p(y=K|\mathbf{x})}&=&\mathbf{w}_1^\mathrm{T}\mathbf{x}+b_1\\
\ln\frac{p(y=2|\mathbf{x})}{p(y=K|\mathbf{x})}&=&\mathbf{w}_2^\mathrm{T}\mathbf{x}+b_2\\
&\dots&\\
\ln\frac{p(y={K-1}|\mathbf{x})}{p(y=K|\mathbf{x})}&=&\mathbf{w}_{K-1}^\mathrm{T}\mathbf{x}+b_{K-1}
\end{eqnarray*}

提示2:定义指示函数$\mathbb{I}(\cdot)$,
$$\mathbb{I}(y=j)=
\begin{cases}
1& \text{若$y$等于$j$}\\
0& \text{若$y$不等于$j$}
\end{cases}$$

\begin{solution}
此处用于写解答(中英文均可)\\
$\because \ln\frac{p(y=i|\mathbf x)}{p(y=K|\mathbf x)}=\mathbf w^T_i \mathbf x + b_i$,$i=1,...,K-1$\\
$\therefore p(y=i|\mathbf x)=p(y=K|\mathbf x) e^{\mathbf w^T_i \mathbf x + b_i}$,$i=1,...,K-1$\\
$\therefore 1=\sum_{i=1}^{K}p(y=i|\mathbf x) = p(y=K|\mathbf x)(\sum_{i=1}^{K-1} e^{\mathbf w^T_i \mathbf x + b_i} + 1)$\\
$\therefore p(y=K|\mathbf x) = \frac{1}{\sum_{l=1}^{K-1} e^{\mathbf w^T_l \mathbf x + b_l} + 1}$\\
$\therefore p(y=i|\mathbf x) = \frac{e^{\mathbf w^T_i \mathbf x + b_i}}{\sum_{l=1}^{K-1} e^{\mathbf w^T_l \mathbf x + b_l} + 1}$,$i=1...K-1$\\
令$\beta_i = (\mathbf w_i;b_i)$,$\hat{\mathbf x}=(\mathbf x;1)$,则$\mathbf w_i^T \mathbf x + b_i = {\beta_i}^T\hat{\mathbf x}$\\
再令$p_i(\hat{\mathbf x};\beta_i)=p(y=i|\hat{\mathbf x};\beta_i)$,设样本数为m\\
$\therefore p(\hat{\mathbf x})=\prod_{i=1}^{K} p_i(\hat{\mathbf x};\beta_i)^{\mathbb{I}(y=i)}$\\
$\therefore L(\beta)=\prod_{j=1}^{m}\prod_{i=1}^{K} p_i(\hat{\mathbf x_j};\beta_i)^{\mathbb{I}(y_j=i)}$\\
$\therefore \ln{L(\beta)} = \sum_{j=1}^m \sum_{i=1}^{K} \mathbb{I}(y_j=i) \ln{p_i(\hat{\mathbf x_j}; \beta_i)}$\\
即$\ln{L(\beta)} = \sum_{j=1}^m \sum_{i=1}^{K} \mathbb{I}(y_j=i) \ln{p(y_j=i|\hat{\mathbf x_j};\beta_i)}$\\
下面求对数似然函数的梯度:\\
$\ln{L(\beta)} = \sum_{j=1}^m \sum_{i=1}^{K} \mathbb{I}(y_j=i) \ln{p(y_j=i|\hat{\mathbf x_j};\beta_i)}$\\
$=\sum_{j=1}^m (\sum_{i=1}^{K-1}\mathbb{I}(y_j=i) [\beta^T_i \hat{\mathbf x_j} - \ln(\sum_{l=1}^{K-1}e^{\beta^T_l \hat{\mathbf x_j}} + 1)] + \mathbb{I}(y_j=K)[0-\ln(\sum_{l=1}^{K-1} e^{\beta^T_l \hat{\mathbf x_j}} + 1)])$\\
$=\sum_{j=1}^m (\sum_{i=1}^{K-1}\mathbb{I}(y_j=i)\beta^T_i \hat{\mathbf x_j} - \sum_{i=1}^{K}\mathbb{I}(y_j=i)\ln(\sum_{l=1}^{K-1} e^{\beta^T_l \hat{\mathbf x_j}} + 1))$\\
$\therefore \frac{\partial \ln{L(\beta)}}{\partial \beta_k} = \sum_{j=1}^m (\mathbb{I}(y_j=k)\hat{\mathbf x_j} - \sum_{i=1}^K\mathbb{I}(y_j=i)\frac{\hat{\mathbf x_j}e^{\beta_k^T \hat{\mathbf x_j}}}{\sum_{l=1}^{K-1} e^{\beta^T_l \hat{\mathbf x_j}} + 1}  )$ \\
$ = \sum_{j=1}^m (\mathbb{I}(y_j=k) - \sum_{i=1}^K\mathbb{I}(y_j=i)\frac{e^{\beta_k^T \hat{\mathbf x_j}}}{\sum_{l=1}^{K-1} e^{\beta^T_l \hat{\mathbf x_j}} + 1})\hat{\mathbf x_j}$\\
$ = \sum_{j=1}^m (\mathbb{I}(y_j=k) - p(y_j=k|\hat{\mathbf x_j};\beta_k))\hat{\mathbf x_j} $
\end{solution}

\section{[35pts] Logistic Regression in Practice} 
对数几率回归(Logistic Regression, 简称LR)是实际应用中非常常用的分类学习算法。

(1) \textbf{[30pts]} 请编程实现二分类的LR, 要求采用牛顿法进行优化求解, 其更新公式可参考《机器学习》教材公式(3.29)。详细编程题指南请参见链接:\url{http://lamda.nju.edu.cn/ml2017/PS2/ML2_programming.html}

(2) \textbf{[5pts]} 请简要谈谈你对本次编程实践的感想(如过程中遇到哪些障碍以及如何解决, 对编程实践作业的建议与意见等)。
\begin{solution}
此处用于写解答(中英文均可)\\
(1)用Python编写的main.py,实现了二分类的LR,自测精度达到95\%以上\\
(2)实验中最大的问题是参数十分敏感。我在一开始测试时,设置$\beta$全为0,精度很好;后来,我将$\beta$初始值设为全1,结果精度立马大幅下降,只剩60\%。这个不经意的改动让我调了很久。我个人的理解是,当$\beta$初值为0时,模型正好对应的是一个随机的分类器,然后学得的$\beta$就是对原来的随机分类器的改进。如果一开始的$\beta$不等于0,则该模型的初始状态就是有偏的,除非迭代次数足够多,否则这个偏离难以被纠正回来。\\
另一方面,因为涉及很多的迭代计算,存在数值不稳定的情况,例如矩阵奇异,如果某次产生了奇异矩阵,那我会跳过当前循环,继续迭代,在后面把奇异矩阵消除。\\
具体编程实现时,还参考了一些\href{https://github.com/deerishi/Logistic-Regression-Convergence-Analysis/blob/master/logistic2.py}{网上的资料}。
\end{solution}


\section{[35pts] Linear Regression with Regularization Term}

给定数据集$D = \{(\mathbf{x}_1,y_1),(\mathbf{x}_2,y_2),\cdots,(\mathbf{x}_m,y_m)\}$, 其中$\mathbf{x}_i = (x_{i1};x_{i2};\cdots;x_{id}) \in \mathbb{R}^d$, $y_i \in \mathbb{R}$, 当我们采用线性回归模型求解时, 实际上是在求解下述优化问题:
\begin{equation}
\label{eq:ls}
\hat{\mathbf{w}}_{\textbf{LS}}^* = \mathop{\arg\min}_{\mathbf{w}} \frac{1}{2}\lVert \mathbf{y} - \mathbf {X}\mathbf{w} \rVert_2^2,
\end{equation}
其中, $\mathbf{y} = [y_1,\cdots,y_m]^\mathrm{T} \in \mathbb{R}^m, \mathbf{X} = [\mathbf{x}_1^\mathrm{T};\mathbf{x}_2^\mathrm{T};\cdots;\mathbf{x}_m^\mathrm{T}]\in \mathbb{R}^{m\times d}$, 下面的问题中, 为简化求解过程, 我们暂不考虑线性回归中的截距(intercept)。

在实际问题中, 我们常常不会直接利用线性回归对数据进行拟合, 这是因为当样本特征很多, 而样本数相对较少时, 直接线性回归很容易陷入过拟合。为缓解过拟合问题, 常对公式\eqref{eq:ls}引入正则化项, 通常形式如下:

\begin{equation}
\label{eq:ls-regular}
\hat{\mathbf{w}}_{\textbf{reg}}^* = \mathop{\arg\min}_{\mathbf{w}} \frac{1}{2}\lVert \mathbf{y} - \mathbf X \mathbf{w} \rVert_2^2 +\lambda \Omega(\mathbf{w}),
\end{equation}
其中, $\lambda> 0$为正则化参数, $\Omega(\mathbf{w})$是正则化项, 根据模型偏好选择不同的$\Omega$。

下面, 假设样本特征矩阵$\mathbf{X}$满足列正交性质, 即$\mathbf{X}^\mathrm{T}\mathbf{X} = \mathbf{I}$, 其中$\mathbf{I}\in \mathbb{R}^{d\times d}$是单位矩阵, 请回答下面的问题(需要给出详细的求解过程):

(1) \textbf{[5pts]} 考虑线性回归问题, 即对应于公式\eqref{eq:ls}, 请给出最优解$\hat{\mathbf{w}}_{\textbf{LS}}^*$的闭式解表达式;

(2) \textbf{[10pts]} 考虑\href{https://en.wikipedia.org/wiki/Tikhonov_regularization}{岭回归(ridge regression)}问题, 即对应于公式\eqref{eq:ls-regular}中$\Omega(\mathbf{w}) = \lVert \mathbf{w}\rVert_2^2=\sum_{i=1}^d w_i^2$时, 请给出最优解$\hat{\mathbf{w}}_{\textbf{Ridge}}^*$的闭式解表达式;

(3) \textbf{[10pts]} 考虑\href{https://en.wikipedia.org/wiki/LASSO}{LASSO}问题, 即对应于公式\eqref{eq:ls-regular}中$\Omega(\mathbf{w}) = \lVert \mathbf{w}\rVert_1=\sum_{i=1}^d \vert w_i\vert$时, 请给出最优解$\hat{\mathbf{w}}_{\textbf{LASSO}}^*$的闭式解表达式;

(4) \textbf{[10pts]} 考虑$\ell_0$-范数正则化问题, 
\begin{equation}
\label{eq:ls-l0}
\hat{\mathbf{w}}_{\mathbf{\ell_0}}^* = \mathop{\arg\min}_{\mathbf{w}} \frac{1}{2}\lVert \mathbf{y} - \mathbf X \mathbf{w} \rVert_2^2 +\lambda \lVert \mathbf{w}\rVert_0,
\end{equation}
其中, $\lVert \mathbf{w}\rVert_0=\sum_{i=1}^d \mathbb{I}[w_i \neq 0]$,即$\lVert \mathbf{w}\rVert_0$表示$\mathbf{w}$中非零项的个数。通常来说, 上述问题是NP-Hard问题, 且是非凸问题, 很难进行有效地优化得到最优解。实际上, 问题(3)中的LASSO可以视为是近些年研究者求解$\ell_0$-范数正则化的凸松弛问题。

但当假设样本特征矩阵$\mathbf{X}$满足列正交性质, 即$\mathbf{X}^\mathrm{T}\mathbf{X} = \mathbf{I}$时, $\ell_0$-范数正则化问题存在闭式解。请给出最优解$\hat{\mathbf{w}}_{\mathbf{\ell_0}}^*$的闭式解表达式, 并简要说明若去除列正交性质假设后, 为什么问题会变得非常困难?

\begin{solution}
此处用于写解答(中英文均可)\\
(1)令$E_{\mathbf w} = \frac{1}{2}\lVert \mathbf y - \mathbf X \mathbf w \rVert_2^2 = \frac{1}{2}(\mathbf y - \mathbf X \mathbf w)^T(\mathbf y - \mathbf X \mathbf w)$\\
则$\frac{\partial E_{\mathbf w}}{\partial w} = \frac{1}{2} 2 \mathbf{X}^T (\mathbf X \mathbf w - \mathbf y) = \mathbf{X}^T (\mathbf X \mathbf w - \mathbf y) = 0$\\
得 $\mathbf{X}^T\mathbf X \mathbf w = \mathbf{X}^T\mathbf y$\\
$\because \mathbf{X}^T\mathbf X = \mathbf I$\\
$\therefore \hat{\mathbf w}^*_{\textbf LS} = (\mathbf{X}^T\mathbf X)^{-1}\mathbf{X}^T\mathbf y = \mathbf{X}^T\mathbf y$\\
(2)令$E_{\mathbf w} = \frac{1}{2}\lVert \mathbf y - \mathbf X \mathbf w \rVert_2^2 + \lambda \lVert \mathbf w\rVert_2^2= \frac{1}{2}(\mathbf y - \mathbf X \mathbf w)^T(\mathbf y - \mathbf X \mathbf w) + \lambda \mathbf{w}^T\mathbf w$\\
则$\frac{\partial E_{\mathbf w}}{\partial w} = \frac{1}{2} 2 \mathbf{X}^T (\mathbf X \mathbf w - \mathbf y) + \lambda 2 \mathbf w= \mathbf{X}^T (\mathbf X \mathbf w - \mathbf y) + 2\lambda \mathbf w=0$\\
得$(\mathbf{X}^T\mathbf X + 2\lambda \mathbf I)\mathbf w = \mathbf{X}^T\mathbf y$\\
$\because \mathbf{X}^T\mathbf X = \mathbf I$\\
$\therefore \hat{\mathbf w}^*_{\textbf Ridge} = \frac{1}{2\lambda + 1}\mathbf{X}^T\mathbf y$\\
(3)根据\href{http://freemind.pluskid.org/machine-learning/sparsity-and-some-basics-of-l1-regularization/}{网上的资料},对于不光滑的$\lVert \mathbf w \rVert_1$,它的梯度不存在,这时需要考虑次梯度(subgradient)\\
点$\mathbf x_0$是凸函数$\mathnormal{f}$的一个全局最小值点,当且仅当$0 \in \partial\mathnormal{f}(\mathbf x_0)$。\\
次梯度的定义为:$$\partial \vert \mathbf w_i\vert =
\begin{cases}
1, \qquad \mathbf w_i > 0  \\
c, \qquad -1\leq c \leq 1, \qquad \mathbf w_i = 0 \\
-1, \qquad \mathbf w_i < 0
\end{cases}$$
令$E_{\mathbf w} = \frac{1}{2}\lVert \mathbf y - \mathbf X \mathbf w \rVert_2^2 + \lambda \lVert \mathbf w\rVert_1= \frac{1}{2}(\mathbf y - \mathbf X \mathbf w)^T(\mathbf y - \mathbf X \mathbf w) + \lambda \sum_{i=1}^d \vert w_i \vert$\\
设$\hat{\mathbf w}^*$为最优解,若梯度存在,此时$\hat{w}_i^* \neq 0$时,此时次梯度等于梯度。\\
$\therefore \frac{\partial E_{\mathbf w}}{\partial \mathbf w_i} = \mathbf{X}^T (\mathbf X \mathbf w - \mathbf y)_i + \lambda sign(w_i) = 0$\\
$\because \mathbf{X}^T\mathbf X = \mathbf I$,且由于极值点处梯度为0\\
$\therefore \hat{w}_i^* = (\mathbf{X}^T \mathbf y)_i - \lambda sign(\hat{w}_i^*) = \hat{w}_{LS,i}^* - \lambda sign(\hat{w}_i^*)$\\
若梯度不存在,则此时$\hat{w}_i^* = 0$, 此时$0 \in \partial \mathbf E_{\mathbf w}$\\
$\therefore \partial \mathbf E_{\mathbf w} \vert_{\hat{w}_i^*} = \hat{w}_i^* - \hat{w}^*_{LS,i} + \lambda c = 0$, 其中,$c=\partial \vert w_i\vert, -1\leq c\leq 1$\\
$\therefore \hat{w}_{LS,i}^* = \lambda c$\\
$$\therefore \hat{w}_{LASSO,i}^* = 
\begin{cases}
\hat{w}^*_{LS,i}-\lambda sign(\hat{w}^*_{LASSO,i}), \qquad gradient\; exists\\
0, \qquad gradient\; does\; not\; exist
\end{cases}$$\\
(4)令$E_{\mathbf w} = \frac{1}{2}\lVert \mathbf y - \mathbf X \mathbf w \rVert_2^2 + \lambda \lVert \mathbf w\rVert_0= \frac{1}{2}(\mathbf y - \mathbf X \mathbf w)^T(\mathbf y - \mathbf X \mathbf w) + \lambda \sum_{i=1}^d \mathbb{I}[w_i \neq 0]$\\
$=\frac{1}{2}\mathbf{y}^T\mathbf y - \mathbf{y}^T\mathbf X \mathbf w + \frac{1}{2}\mathbf{w}^T\mathbf w + \lambda \sum_{i=1}^d \mathbb{I}[w_i \neq 0]$\\
由于$\mathbf{y}^T\mathbf y$为常量,所以,问题转化为求:\\
$\textbf{min} -\mathbf{y}^T\mathbf X \mathbf w + \frac{1}{2}\sum_{i=1}^{d} w_i^2 + \lambda \sum_{i=1}^d \mathbb{I}[w_i \neq 0]$\\
又因为$\mathbf{X}^T \mathbf y = \hat{\mathbf w}^*_{\textbf LS}$\\
$\therefore$原问题转化为:\\
$\textbf{min} \quad \mathbf{\mathnormal{f}}=\sum_{i=1}^d -\hat{w}^*_{i,LS} w_i + \frac{1}{2}\sum_{i=1}^d w_i^2 + \lambda \sum_{i=1}^d \mathbb{I}[w_i \neq 0]$\\
当$w_i \neq 0$时,$\frac{\partial \mathbf{\mathnormal{f}}}{\partial w_i} = -\hat{w}^*_{i,LS} + w_i = 0$\\
$\therefore \hat{w}_{i,\ell_0}^* = \hat{w}^*_{i,LS}$\\
当$w_i=0$时,$\hat{w}_{i,\ell_0}^* = 0$\\
去掉正交性条件问题就会复杂主要是因为$\hat{\mathbf w}^*_{\textbf LS} = (\mathbf{X}^T\mathbf X)^{-1}\mathbf{X}^T\mathbf y = \mathbf{X}^T\mathbf y$不成立了,使得问题无法简化成上述形式。
\end{solution}



\end{document}