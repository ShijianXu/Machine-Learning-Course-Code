\documentclass[a4paper,UTF8]{article}
\usepackage{ctex}
\usepackage[margin=1.25in]{geometry}
\usepackage{color}
\usepackage{graphicx}
\usepackage{amssymb}
\usepackage{amsmath}
\usepackage{amsthm}
\usepackage{enumerate}
\usepackage{bm}
\usepackage{hyperref}
\numberwithin{equation}{section}
%\usepackage[thmmarks, amsmath, thref]{ntheorem}
\theoremstyle{definition}
\newtheorem*{solution}{Solution}
\newtheorem*{prove}{Proof}
\newcommand{\indep}{\rotatebox[origin=c]{90}{$\models$}}

\usepackage{multirow}

%--

%--
\begin{document}
\title{机器学习导论\\
习题六}
\author{141220120, 徐世坚, xsj13260906215@gmail.com}
\maketitle

\section{[20pts] Ensemble Methods}
\begin{enumerate}[ {(}1{)}]
\item \textbf{[10pts]} 试说明Boosting的核心思想是什么,Boosting中什么操作使得基分类器具备多样性?

\item \textbf{[10pts]} 试析随机森林为何比决策树Bagging集成的训练速度更快。
\end{enumerate}
\begin{solution}
此处用于写解答(中英文均可)\\
(1)Boosting的核心思想是先基于原始数据集训练出一个基学习器,然后根据这个学习器的表现对训练集样本分布进行调整,使得先前做错的训练样本在后续得到更高的关注,然后基于调整后的数据集训练下一个基学习器,迭代进行下去。\\
Boosting中对训练集样本分布的调整使得基分类器具备多样性。\\
(2)在决策树Bagging集成中,每次选择属性划分需要考察结点所有的属性,而随机森林在每个结点上,只需要随机考察一个属性子集。所以随机森林的训练速度更快。\\
\end{solution}

\section{[20pts] Bagging}
考虑一个回归学习任务$f:\mathbb{R}^d \rightarrow \mathbb{R}$。假设我们已经学得$M$个学习器$\hat{f}_1(\mathbf{x}),\hat{f}_2(\mathbf{x}),\dots,\hat{f}_M(\mathbf{x})$。我们可以将学习器的预测值看作真实值项加上误差项
\begin{equation}
\hat{f}_m(\mathbf{x})=f(\mathbf{x})+\epsilon_m(\mathbf{x})
\end{equation}
每个学习器的期望平方误差为$\mathbb{E}_{\mathbf{x}}[\epsilon_m(\mathbf{x})^2]$。所有的学习器的期望平方误差的平均值为
\begin{equation}
E_{av}=\frac{1}{M}\sum_{m=1}^M \mathbb{E}_{\mathbf{x}}[\epsilon_m(\mathbf{x})^2]
\end{equation}
M个学习器得到的Bagging模型为
\begin{equation}
\hat{f}_{bag}(\mathbf{x})=\frac{1}{M}\sum_{m=1}^M \hat{f}_m(\mathbf{x})
\end{equation}
Bagging模型的误差为
\begin{equation}
\epsilon_{bag}(\mathbf{x})=\hat{f}_{bag}(\mathbf{x})-f(\mathbf{x})=\frac{1}{M}\sum_{m=1}^M \epsilon_m(\mathbf{x})
\end{equation}
其期望平均误差为
\begin{equation}
E_{bag}=\mathbb{E}_{\mathbf{x}}[\epsilon_{bag}(\mathbf{x})^2]
\end{equation}
\begin{enumerate}[ {(}1{)}]
\item \textbf{[10pts]} 假设$\forall\; m\neq l$,$ \mathbb{E}_{\mathbf{x}}[\epsilon_m(\mathbf{x})]=0$,$ \mathbb{E}_{\mathbf{x}}[\epsilon_m(\mathbf{x})\epsilon_l(\mathbf{x})]=0$。证明
\begin{equation}
E_{bag}=\frac{1}{M} E_{av}
\end{equation}

\item  \textbf{[10pts]} 试证明不需对$\epsilon_m(\mathbf{x})$做任何假设,$E_{bag}\leq E_{av}$始终成立。(提示:使用Jensen's inequality)
\end{enumerate}

\begin{prove}
此处用于写证明(中英文均可)\\
(1)
$E_{bag}=\mathbb{E}_{\mathbf{x}}[\epsilon_{bag}(\mathbf{x})^2]$\\
$ \qquad =\mathbb{E}_{\mathbf{x}}[(\frac{1}{M} \sum_{m=1}^M \epsilon_{m}(\mathbf{x}))^2]$\\
$ \qquad =\mathbb{E}_{\mathbf{x}}[\frac{1}{M^2} ( \sum_{m=1}^M \epsilon_m(\mathbf{x}) )^2]$\\
$ \qquad =\frac{1}{M^2} \mathbb{E}_{\mathbf{x}} [ \sum_{m=1}^M \sum_{n=1}^M \epsilon_m(\mathbf{x})\epsilon_n(\mathbf{x}) ]$\\
$ \qquad =\frac{1}{M^2} \sum_{m=1}^M \sum_{n=1}^M \mathbb{E}_{\mathbf{x}} [ \epsilon_m(\mathbf{x})\epsilon_n(\mathbf{x}) ]$\\
$\because \forall\; m\neq l$,$ \mathbb{E}_{\mathbf{x}}[\epsilon_m(\mathbf{x})]=0$,$ \mathbb{E}_{\mathbf{x}}[\epsilon_m(\mathbf{x})\epsilon_l(\mathbf{x})]=0$\\
$\therefore E_{bag} = \frac{1}{M^2} \sum_{m=1}^M \mathbb{E}_{\mathbf{x}} [ \epsilon_m(\mathbf{x})^2 ]=\frac{1}{M} E_{av}$\\
(2)
由Jensen's inequality可知\\
$(\frac{1}{M} \sum_{m=1}^M \epsilon_{\mathbf{m}} (\mathbf{x}))^2 \leq \frac{1}{M}\sum_{m=1}^M \epsilon_{\mathbf{m}} (\mathbf{x})^2$\\
两边同取期望,可得\\
$\mathbb{E}_{\mathbf{x}} [(\frac{1}{M} \sum_{m=1}^M \epsilon_{\mathbf{m}} (\mathbf{x}))^2]  \leq \mathbb{E}_{\mathbf{x}} [\frac{1}{M}\sum_{m=1}^M \epsilon_{\mathbf{m}} (\mathbf{x})^2 ]$\\
$\mathbb{E}_{\mathbf{x}}[\epsilon_{bag}(\mathbf{x})^2] \leq \frac{1}{M} \sum_{m=1}^M \mathbb{E}_{\mathbf{x}} [ \epsilon_m(\mathbf{x})^2 ]$\\
$\therefore E_{bag} \leq E_{av}$
\qed
\end{prove}

\section{[30pts] AdaBoost in Practice}

\begin{enumerate}[ {(}1{)}]
\item \textbf{[25pts]} 请实现以Logistic Regression为基分类器的AdaBoost,观察不同数量的ensemble带来的影响。详细编程题指南请参见链接:
\url{http://lamda.nju.edu.cn/ml2017/PS6/ML6_programming.html}
\item \textbf{[5pts]} 在完成上述实践任务之后,你对AdaBoost算法有什么新的认识吗?请简要谈谈。
\end{enumerate}
\begin{solution}
此处用于写解答(中英文均可)\\
直观感受是,adaboost随着集成数目的增加,精度会提高很多。\\
另外就是,由于在sklearn中的logistic regression 是按如下方式实现的:\\
$\underset{w, c}{min\,} \frac{1}{2}w^T w + C \sum_{i=1}^n \log(\exp(- y_i (X_i^T w + c)) + 1)$\\
其中的C的作用和SVM中的具有相同作用。所以,当适当调高C的值时,单个logistics regression得到的分类器的精度会提高很多,从而导致整体的精度也会提高。\\
至于样本权重的归一化问题,一个现象是,不归一化的权重得到的结果反而比归一化之后的结果更好,这我不是很理解,不知道实现上是怎么处理的。\\
最后就是从未碰到过的坑了——Python的深浅拷贝问题。因为我在类中定义了一个基分类器成员,所以我每次都是用同一个基分类器来训练的,然后将fit得到的模型通过list.append()加进去。但是list.append()函数是浅拷贝,导致最终只有一个模型是有效的。这个错误实在是太难发现了。\\
\end{solution}
\end{document}