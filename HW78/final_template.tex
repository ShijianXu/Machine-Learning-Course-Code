\documentclass[a4paper,UTF8]{article}
\usepackage{ctex}
\usepackage[margin=1.25in]{geometry}
\usepackage{color}
\usepackage{graphicx}
\usepackage{enumerate}
\usepackage{hyperref}
\usepackage{amsfonts, amsmath, amsthm, bm, amssymb}
\numberwithin{equation}{section}
\usepackage[ruled,vlined,lined,boxed,linesnumbered]{algorithm2e}
\theoremstyle{definition}
\newtheorem*{solution}{Solution}
\newtheorem*{prove}{Proof}
\newcommand{\indep}{\rotatebox[origin=c]{90}{$\models$}}
%--

%--
\begin{document}
\title{机器学习导论\\
综合能力测试}
\author{141220120, 徐世坚, xsj13260906215@gmail.com}
\maketitle
\section{[40pts] Exponential Families}
\label{Exponential Families}
指数分布族(\href{https://en.wikipedia.org/wiki/Exponential_family}{Exponential Families})是一类在机器学习和统计中非常常见的分布族, 具有良好的性质。在后文不引起歧义的情况下, 简称为指数族。

指数分布族是一组具有如下形式概率密度函数的分布族群:
\begin{equation}
f_X(x|\theta) = h(x) \exp \left(\eta(\theta) \cdot T(x) -A(\theta)\right)
\end{equation}  
其中, $\eta(\theta)$, $A(\theta)$以及函数$T(\cdot)$, $h(\cdot)$都是已知的。
\begin{enumerate}[(1)]
\item \textbf{[10pts]} 试证明多项分布(\href{https://en.wikipedia.org/wiki/Multinomial_distribution}{Multinomial distribution})属于指数分布族。

\item \textbf{[10pts]} 试证明多元高斯分布(\href{https://en.wikipedia.org/wiki/Multivariate_normal_distribution}{Multivariate Gaussian distribution})属于指数分布族。

\item \textbf{[20pts]} 考虑样本集$\mathcal{D}=\{ x_1,\cdots, x_n\}$是从某个已知的指数族分布中独立同分布地(i.i.d.)采样得到, 即对于$\forall i\in [1,n]$, 我们有$f( x_i|\boldsymbol\theta) = h(x_i) \exp \left ( {\boldsymbol\theta}^{\rm T}T(x_i) -A(\boldsymbol\theta)\right)$. 

对参数$\boldsymbol\theta$, 假设其服从如下先验分布:
\begin{equation}
p_\pi(\boldsymbol\theta|\boldsymbol\chi,\nu) = f(\boldsymbol\chi,\nu) \exp \left (\boldsymbol\theta^{\rm T} \boldsymbol\chi - \nu A(\boldsymbol\theta) \right )
\end{equation}
其中, $\boldsymbol\chi$和$\nu$是$\boldsymbol\theta$生成模型的参数。请计算其后验, 并证明后验与先验具有相同的形式。(\textbf{Hint}: 上述又称为“共轭”(\href{https://people.eecs.berkeley.edu/~jordan/courses/260-spring10/other-readings/chapter9.pdf}{Conjugacy}),在贝叶斯建模中经常用到)
\end{enumerate}

\begin{solution}
此处用于写证明(中英文均可)\\
(1)$P(x_1,x_2,...,x_d | N, \boldsymbol{\mu}) = \frac{N!}{x_1!x_2!...x_d!} \prod_{i=1}^d \mu_i^{x_i}$\\
$= \frac{N!}{x_1!x_2!...x_d!} \exp(\sum_{i=1}^d x_i \ln \mu_i)$\\
$= \frac{1}{x_1!x_2!...x_d!} \exp(\sum_{i=1}^d x_i \ln \mu_i + \ln N!)$\\
令$\theta = (N, \boldsymbol{\mu})$,则\\
$h(\mathbf{x}) = \frac{1}{x_1!x_2!...x_d!}$, $\eta(\theta) = (\ln \boldsymbol{\mu}) = [\ln \mu_1 ,..., \ln \mu_d ]$,$T(\mathbf{x}) = \mathbf{x}$,$A(\theta) = - \ln N!$\\
$\therefore$多项分布属于指数分布族。\\
(2)$P(\mathbf{x} | \boldsymbol{\mu}, \mathbf{\Sigma}) = (2\pi)^{-\frac{d}{2}} |\mathbf{\Sigma}|^{-\frac{1}{2}} \exp(-\frac{1}{2} (\mathbf{x}-\boldsymbol{\mu})^T \mathbf{\Sigma}^{-1} (\mathbf{x}-\boldsymbol{\mu}))$\\
$=(2\pi)^{-\frac{d}{2}} \exp( -\frac{1}{2} (\mathbf{x}-\boldsymbol{\mu})^T \mathbf{\Sigma}^{-1} (\mathbf{x}-\boldsymbol{\mu}) - \frac{1}{2}\ln |\mathbf{\Sigma}| )$\\
$=(2\pi)^{-\frac{d}{2}} \exp( -\frac{1}{2} (\mathbf{x}^T\mathbf{\Sigma}^{-1}\mathbf{x} - \mathbf{x}^T\mathbf{\Sigma}^{-1}\boldsymbol{\mu} - \boldsymbol{\mu}\mathbf{\Sigma}^{-1}\mathbf{x} + \boldsymbol{\mu}^T \mathbf{\Sigma}^{-1}\boldsymbol{\mu} ) - \frac{1}{2}\ln |\mathbf{\Sigma}|)$\\
$=(2\pi)^{-\frac{d}{2}} \exp( -\frac{1}{2}\mathbf{x}^T\mathbf{\Sigma}^{-1}\mathbf{x} + \boldsymbol{\mu}^T \mathbf{\Sigma}^{-1} \mathbf{x} -(\frac{1}{2}\boldsymbol{\mu}^T \mathbf{\Sigma}^{-1}\boldsymbol{\mu}+\frac{1}{2}\ln |\mathbf{\Sigma}|) )$\\
$\therefore h(\mathbf{x}) = (2\pi)^{-\frac{d}{2}}$,$\eta(\theta) = [\Sigma^{-1}\boldsymbol\mu; -\frac{1}{2}\Sigma^{-1}]$,$T(\mathbf{x}) = [\mathbf{x}; \mathbf{x}\mathbf{x}^T]$,$A(\theta) = \frac{1}{2}\boldsymbol\mu^T \Sigma^{-1} \boldsymbol\mu + \frac{1}{2}\ln | \Sigma|$\\
$\therefore$多元高斯分布属于指数分布。\\
(3)$P(\boldsymbol{\theta} | \mathbf{X}) \propto p_{\pi} (\boldsymbol\theta | \boldsymbol\chi, \nu) f(\mathbf{X} | \boldsymbol\theta)$\\
$= f(\boldsymbol\chi,\nu)\exp(\boldsymbol\theta^T\boldsymbol\chi-\nu A(\boldsymbol\theta)) \prod_{i=1}^n f(x_i | \boldsymbol\theta)$\\
$= f(\boldsymbol\chi,\nu)\exp(\boldsymbol\theta^T\boldsymbol\chi-\nu A(\boldsymbol\theta)) \prod_{i=1}^n h(x_i)\exp(\boldsymbol\theta^T T(x_i) - A(\boldsymbol\theta)) $\\
$= f(\boldsymbol\chi,\nu) \prod_{i=1}^n h(x_i) \exp(\boldsymbol\theta^T(\boldsymbol\chi + \sum_{i=1}^n T(x_i) ) -(\nu+n)A(\boldsymbol\theta) )$\\
$\propto \exp(\boldsymbol\theta^T(\boldsymbol\chi + \sum_{i=1}^n T(x_i)) -(\nu+n)A(\boldsymbol\theta)$\\
$\propto p_{\pi} (\boldsymbol\theta | \boldsymbol\chi + \sum_{i=1}^n T(x_i), \nu+n)$\\
$\therefore$后验与先验具有相同的形式。\\
\end{solution}

\newpage
\section{[40pts] Decision Boundary}
考虑二分类问题, 特征空间$X \in \mathcal{X}= \mathbb{R}^d$, 标记$Y \in \mathcal{Y}= \{0, 1\}$. 我们对模型做如下生成式假设:
\begin{itemize}
\item[-] attribute conditional independence assumption: 对已知类别, 假设所有属性相互独立, 即每个属性特征独立地对分类结果发生影响;
\item[-] Bernoulli prior on label: 假设标记满足Bernoulli分布先验, 并记$\Pr(Y=1) = \pi$. 
\end{itemize}

\begin{enumerate}[(1)]
\item \textbf{[20pts]} 假设$P(X_i | Y)$服从指数族分布, 即
\[
\Pr(X_i = x_i | Y = y) = h_i(x_i) \exp (\theta_{iy} \cdot T_i(x_i) - A_{i}(\theta_{iy}))
\]
请计算后验概率分布$\Pr(Y | X)$以及分类边界$\{x \in \mathcal{X}: P(Y=1 | X = x) = P(Y=0 | X =x)\}$. (\textbf{Hint}: 你可以使用sigmoid函数$\mathcal{S}(x)=1/(1+e^{-x})$进行化简最终的结果).

\item \textbf{[20pts]} 假设$P(X_i | Y=y)$服从高斯分布, 且记均值为$\mu_{iy}$以及方差为$\sigma_{i}^2$ (注意, 这里的方差与标记$Y$是独立的), 请证明分类边界与特征$X$是成线性的。 
\end{enumerate}
\begin{solution}
此处用于写解答(中英文均可)\\
(1)$Pr(Y=1|X=x) = \frac{Pr(X=x|Y=1)Pr(Y=1)}{Pr(X=x|Y=1)Pr(Y=1) + Pr(X=x|Y=0)Pr(Y=0)}$\\
$= \frac{ \pi \prod_{i=1}^d  h_i(x_i)\exp(\theta_{i1}T_i(x_i)-A_i(\theta_{i1})) }{\pi \prod_{i=1}^d  h_i(x_i)\exp(\theta_{i1}T_i(x_i)-A_i(\theta_{i1})) +  (1-\pi)\prod_{i=1}^d  h_i(x_i)\exp(\theta_{i0}T_i(x_i)-A_i(\theta_{i0})) }$\\
$= \frac{1}{ 1+\exp( \sum_{i=1}^dT_i(x_i)(\theta_{i0}-\theta_{i1})-\sum_{i=1}^d( A_i(\theta_{i0})-A_i(\theta_{i1}) ) + \ln(1-\pi) - \ln\pi ) }$\\
$= S(\sum_{i=1}^d( A_i(\theta_{i0})-A_i(\theta_{i1}) ) - \sum_{i=1}^dT_i(x_i)(\theta_{i0}-\theta_{i1}) - (\ln(1-\pi) - \ln\pi))$\\
同理可得:\\
$Pr(Y=0|X=x) = S(\sum_{i=1}^d( A_i(\theta_{i1})-A_i(\theta_{i0}) ) - \sum_{i=1}^dT_i(x_i)(\theta_{i1}-\theta_{i0}) - (\ln\pi - \ln(1-\pi) ))$\\
为求分类边界,令$Pr(Y=1|X=x) = Pr(Y=0|X=x)$,得:\\
$\pi\prod_{i=1}^d \exp(\theta_{i1}T_i(x_i)-A_i(\theta_{i1})) = (1-\pi)\prod_{i=1}^d \exp(\theta_{i0}T_i(x_i)-A_i(\theta_{i0}))$\\
两边同取$\ln$,得:\\
$\sum_{i=1}^d ( \theta_{i1}T_i(x_i)-A_i(\theta_{i1}) ) + \ln\pi = \sum_{i=1}^d ( \theta_{i0}T_i(x_i)-A_i(\theta_{i0}) ) + \ln(1-\pi)$\\
$\sum_{i=1}^d ( \theta_{i1}-\theta_{i0} )T_i(x_i) = \ln \frac{1-\pi}{\pi} + \sum_{i=1}^d ( A_i(\theta_{i1})-A_i(\theta_{i0}) )$\\
$\therefore$上式可以写成$\mathbf{w^T}T(\mathbf{x}) + b = 0$\\
(2)令$Pr(Y=1|X=x) = Pr(Y=0|X=x)$,得:\\
$\pi \prod_{i=1}^d (\frac{1}{\sqrt{2\pi}\sigma_i} \exp(-\frac{(x_i-\mu_{i1})^2}{2\sigma_i^2})) = (1-\pi) \prod_{i=1}^d (\frac{1}{\sqrt{2\pi}\sigma_i} \exp(-\frac{(x_i-\mu_{i0})^2}{2\sigma_i^2}))$\\
$\ln \frac{\pi}{1-\pi} + \sum_{i=1}^d [\ln\frac{1}{\sqrt{2\pi}\sigma_i} - \frac{(x_i - \mu_{i1})^2}{2\sigma_i^2}] = \sum_{i=1}^d [\ln\frac{1}{\sqrt{2\pi}\sigma_i} - \frac{(x_i - \mu_{i0})^2}{2\sigma_i^2}]$\\
$\ln \frac{\pi}{1-\pi} = \sum_{i=1}^d [ \frac{(x_i - \mu_{i1})^2}{2\sigma_i^2} - \frac{(x_i - \mu_{i0})^2}{2\sigma_i^2} ]$\\
$\sum_{i=1}^d \frac{\mu_{i0}-\mu_{i1}}{\sigma_i^2} x_i - \sum_{i=1}^d \frac{\mu_{i0}^2-\mu_{i1}^2}{2\sigma_i^2} - \ln\frac{\pi}{1-\pi}=0$\\
$\therefore$上式可以写成$\mathbf{w^T}\mathbf{x} + b = 0$\\
$\therefore$分类边界和$X$成线性关系。
\end{solution}



\newpage
\section{[70pts] Theoretical Analysis of $k$-means Algorithm}
给定样本集$\mathcal{D} = \{ \mathbf x_1,\mathbf x_2, \ldots, \mathbf x_n \}$, $k$-means聚类算法希望获得簇划分$\mathcal{C}=\{C_1,C_2,\cdots,C_k\}$, 使得最小化欧式距离
\begin{equation}
\label{eq-kmeans-l2}
J(\gamma, \mu_1,\ldots,\mu_k) = \sum_{i=1}^n \sum_{j=1}^k \gamma_{ij}||\mathbf x_i - \mu_j||^2
\end{equation} 
其中, $\mu_1, \ldots, \mu_k$为$k$个簇的中心(means), $\gamma \in \mathbb{R}^{n\times k}$为指示矩阵(indicator matrix)定义如下:若$\mathbf x_i$属于第$j$个簇, 则$\gamma_{ij} = 1$, 否则为0. 

则最经典的$k$-means聚类算法流程如算法\ref{algo:kmeans}中所示(与课本中描述稍有差别, 但实际上是等价的)。
\begin{algorithm}[]
\label{algo:kmeans}
\caption{$k$-means Algorithm}
\setcounter{AlgoLine}{0}
Initialize $\mu_1, \ldots, \mu_k$.\\
\Repeat{the objective function $J$ no longer changes}{
\textbf{Step 1}: Decide the class memberships of $\{\mathbf x_i\}_{i=1}^n$ by assigning each of them to its nearest cluster center.
\begin{align*}
\gamma_{ij} =
\begin{cases} 
1,& ||\mathbf x_i - \mu_j||^2 \le ||\mathbf x_i - \mu_{j'}||^2, \forall j' \\
0, & \text{otherwise} 
\end{cases}
\end{align*}\\
\textbf{Step 2}: For each $j \in \{1, \cdots, k\}$, recompute $\mu_j$ using the updated $\gamma$ to be the center of mass of all points in $C_j$: 
\begin{align*}
\mu_j = \frac{\sum_{i=1}^n \gamma_{ij}\mathbf x_i}{\sum_{i=1}^n \gamma_{ij}}
\end{align*}
}
\end{algorithm}

\begin{enumerate}[(1)]

\item \textbf{[10pts]} 试证明, 在算法\ref{algo:kmeans}中, \textbf{Step 1}和\textbf{Step 2}都会使目标函数$J$的值降低.

\item \textbf{[10pts]} 试证明, 算法\ref{algo:kmeans}会在有限步内停止。

\item {\textbf{[10pts]} 试证明, 目标函数$J$的最小值是关于$k$的非增函数, 其中$k$是聚类簇的数目。}

\item {\textbf{[20pts]} 记$\hat{\mathbf{x}}$为$n$个样本的中心点, 定义如下变量,
\begin{table}[h]
\centering
\label{table:equation}
\begin{tabular}{ l | c }
  \hline			
total deviation & $T(X) = \sum_{i=1}^n \lVert \mathbf x_i - \hat{\mathbf x}\rVert^2/n$ \\
intra-cluster deviation & $W_j(X) = \sum_{i=1}^n \gamma_{ij} \lVert\mathbf x_i - \mu_j \rVert^2/\sum_{i=1}^n \gamma_{ij}$ \\
inter-cluster deviation & $B(X) = \sum_{j=1}^k \frac{ \sum_{i=1}^n \gamma_{ij}}{n}  \lVert\mu_j -\hat{\mathbf x} \rVert^2$\\
  \hline  
\end{tabular}
\end{table}

试探究以上三个变量之间有什么样的等式关系?基于此, 请证明, $k$-means聚类算法可以认为是在最小化intra-cluster deviation的加权平均, 同时近似最大化inter-cluster deviation.}

\item { \textbf{[20pts]} 在公式\eqref{eq-kmeans-l2}中, 我们使用$\ell_2$-范数来度量距离(即欧式距离), 下面我们考虑使用$\ell_1$-范数来度量距离

\begin{equation}
\label{eq-kmeans-l1}
J'(\gamma, \mu_1,\ldots,\mu_k) = \sum_{i=1}^n \sum_{j=1}^k \gamma_{ij}||\mathbf x_i - \mu_j||_1
\end{equation}

\begin{itemize} 
\item \textbf{[10pts]} 请仿效算法\ref{algo:kmeans}($k$-means-$\ell_2$算法), 给出新的算法(命名为$k$-means-$\ell_1$算法)以优化公式\ref{eq-kmeans-l1}中的目标函数$J'$.
\item \textbf{[10pts]} 当样本集中存在少量异常点(\href{https://en.wikipedia.org/wiki/Outlier}{outliers})时, 上述的$k$-means-$\ell_2$和$k$-means-$\ell_1$算法, 我们应该采用哪种算法?即, 哪个算法具有更好的鲁棒性?请说明理由。
\end{itemize}}

\end{enumerate}

\begin{solution}
此处用于写解答(中英文均可)\\
(1)对任意的$\mathbf{x}_i$,设它原来属于第$\lambda_i$类,而在Step1中修改为第$\lambda_i^{'}$类。则\\
$J^{'}(\gamma, \boldsymbol\mu_1,...,\boldsymbol\mu_k) = \sum_{i=1}^n \sum_{j=1}^k \gamma_{ij}^{'} ||\mathbf{x}_i - \boldsymbol \mu_j ||^2$\\
$ = \sum_{i=1}^n ||\mathbf{x}_i -\boldsymbol\mu_{\lambda_i^{'}}||^2$\\
$ \le \sum_{i=1}^n || \mathbf{x}_i - \boldsymbol\mu_{\lambda_i} ||^2$\\
$ = \sum_{i=1}^n \sum_{j^=1}^k \gamma_{ij}|| \mathbf{x}_i - \boldsymbol\mu_j ||^2$\\
$ = J(\gamma, \boldsymbol\mu_1,...,\boldsymbol\mu_k)$\\
$\therefore$Step1使得目标函数$J$的值降低(非增)。\\
对于Step2,从$J$的表达式可知,它计算的是所有类的类内平方距离的和。所以考虑任意一个类$C_j$,它的类内平方距离为:\\
$\sum_{\mathbf{x}_i\in C_j} || \mathbf{x}_i - \boldsymbol\mu_j ||^2$\\
为了使平方距离最小,对$\boldsymbol\mu_j$进行求导:\\
$\frac{\partial \sum_{\mathbf{x}_i\in C_j} || \mathbf{x}_i - \boldsymbol\mu_j ||^2}{\partial \boldsymbol\mu_j}$\\
令偏导为0,可得:\\
$\sum_{\mathbf{x}_i \in C_j} \mathbf{x_i} = \sum_{\mathbf{x}_i \in C_j} \boldsymbol\mu_j$\\
得:$\boldsymbol\mu_j = \frac{1}{|C_j|} \sum_{\mathbf{x}_i\in C_j} \mathbf{x}_i = \frac{\sum_{i=1}^n \gamma_{ij}\mathbf{x}_i}{\sum_{i=1}^n \gamma_{ij}}$\\
即Step2中的调整就是最优的解\\
$\therefore$Step2使得目标函数$J$的值降低(非增)。\\
(2)因为有n个样本,k个类别,所以所有的可能的划分个数为$k^n$个。\\
算法每一轮迭代,如果做了调整,那么一定产生的是一个新的划分,该划分对应的目标函数比之前的都要小。而如果该轮迭代没有做调整,则目标函数的值不变,算法终止。\\
当算法终止时,所遍历的划分个数一定是一个有限值,且小于$k^n$。所以,算法会在有限步内停止。\\
(3)假设当前有k类,且算法已经停止,即当前的J的值为最小值。则当增加一个新的类时(增加一个新的$\boldsymbol\mu_{k+1}$),算法会继续进行。\\
如果在Step1和Step2中没有发生变动,则目标函数J的值将不变。而如果Step1和Step2有进行调整,则由前面的结论可知,目标函数J的值将会降低。这样继续迭代得到新的最小值。\\
所以目标函数J的最小值是关于k的非增函数。同时可发现,当$k=n$时,$J$的值最小,为0,即每个样例自成一类,但此时的分类无意义。\\
(4)$\sum_{j=1}^k\sum_{i=1}^n\gamma_{ij}W_j(\mathbf{X}) + nB(\mathbf{X})$\\
$=\sum_{j=1}^k\sum_{i=1}^n\gamma_{ij} ( || \mathbf{x}_i - \boldsymbol\mu_j ||^2 + || \boldsymbol\mu_j - \hat{\mathbf{x}} ||^2 )$\\
$=\sum_{j=1}^k\sum_{i=1}^n\gamma_{ij} ( \mathbf{x}_i^T\mathbf{x}_i -2\mathbf{x}_i^T\boldsymbol\mu_j + 2\boldsymbol\mu_j^T\boldsymbol\mu_j - 2\boldsymbol\mu_j^T\hat{\mathbf{x}} + \mathbf{x}^T\mathbf{x})$\\
$=\sum_{j=1}^k\sum_{i=1}^n\gamma_{ij} ( \mathbf{x}_i^T\mathbf{x}_i -2\mathbf{x}_i\hat{\mathbf{x}} +2\mathbf{x}_i\hat{\mathbf{x}}-2\mathbf{x}_i^T\boldsymbol\mu_j + 2\boldsymbol\mu_j^T\boldsymbol\mu_j - 2\boldsymbol\mu_j^T\hat{\mathbf{x}} + \mathbf{x}^T\mathbf{x} )$\\
$=\sum_{j=1}^k\sum_{i=1}^n\gamma_{ij} || \mathbf{x}_i -\hat{\mathbf{x}} ||^2 + \sum_{j=1}^k\sum_{i=1}^n\gamma_{ij} 2 ( \mathbf{x}_i\hat{\mathbf{x}} - \mathbf{x}_i^T\boldsymbol\mu_j + \boldsymbol\mu_j^T\boldsymbol\mu_j - \boldsymbol\mu_j^T\hat{\mathbf{x}})$\\
=$nT(X) + \sum_{j=1}^k\sum_{i=1}^n\gamma_{ij} 2 ( \mathbf{x}_i\hat{\mathbf{x}} - \mathbf{x}_i^T\boldsymbol\mu_j + \boldsymbol\mu_j^T\boldsymbol\mu_j - \boldsymbol\mu_j^T\hat{\mathbf{x}})$\\
因为$\sum_{j=1}^k\sum_{i=1}^n\gamma_{ij}W_j(\mathbf{X}) = J$,而如果右边近似看成常数的话($n$和$T(X)$均为常数,变化的是最右边的求和部分),则在最小化intra-cluster deviation的加权平均(即目标函数$J$)时,相应的$nB(\mathbf{X})$会增大,即近似最大化inter-cluster deviation.\\
(5)$k-means-\ell_1$算法如下:\\
\begin{algorithm}[]
\label{algo:kmeans}
\caption{$k$-means-$\ell_1$ Algorithm}
\setcounter{AlgoLine}{0}
Initialize $\mu_1, \ldots, \mu_k$.\\
\Repeat{the objective function $J$ no longer changes}{
\textbf{Step 1}: Decide the class memberships of $\{\mathbf x_i\}_{i=1}^n$ by assigning each of them to its nearest cluster center.
\begin{align*}
\gamma_{ij} =
\begin{cases} 
1,& ||\mathbf x_i - \mu_j||_1 \le ||\mathbf x_i - \mu_{j'}||_1, \forall j' \\
0, & \text{otherwise} 
\end{cases}
\end{align*}\\
\textbf{Step 2}: For each $j \in \{1, \cdots, k\}$, recompute $\mu_j$ using the updated $\gamma$ to be the center of mass of all points in $C_j$: 
\begin{align*}
\mu_j = the\quad median\quad of\quad all\quad x_i \in Cluster_j
\end{align*}
}
\end{algorithm}\\
应该采用$k$-means-$\ell_1$算法。考虑一个类不幸分到了一个或多个异常点,则在计算簇的中心$\boldsymbol\mu$时,如果采用$\ell_2$范数来度量距离,则计算得到的$\boldsymbol\mu$离最优的中心会产生较大的偏差,这些偏差在后面的迭代中会累积,最终导致分类结果很差;而如果采用$\ell_1$范数来度量距离,则对中心的计算影响很小甚至没有影响,因为这时的中心是类中样本的中位数,距离对中心的计算影响不大,这样的话最终的效果会好很多。\\
\end{solution}


\newpage
\section{[50pts] Kernel, Optimization and Learning}
给定样本集$\mathcal{D} = \{(\mathbf{x}_1,y_1),(\mathbf{x}_2,y_2),\cdots,(\mathbf{x}_m,y_m)\}$, $\mathcal{F} = \{\Phi_1 \cdots,\bm \Phi_d\}$为非线性映射族。考虑如下的优化问题
\begin{equation}
\label{eq-primal}
\min_{\mathbf w, \mu\in \Delta_q} \quad \frac{1}{2} \sum_{k=1}^d \frac{1}{\mu_k}\lVert\mathbf w_k\rVert_2^2 + C\sum_{i=1}^m \max\left\lbrace 0,1 - y_i\left(\sum_{k=1}^d \mathbf w_k \cdot \bm \Phi_k(\mathbf{x}_i)\right) \right\rbrace
\end{equation}
其中, $\Delta_q = \left\lbrace \bm{ \mu} | \mu_k\geq 0, k=1,\cdots,d; \lVert \bm{ \mu} \rVert_q = 1\right\rbrace$.

\begin{enumerate}[(1)]
\item{ \textbf{[30pts]} 请证明, 下面的问题\ref{eq-dual}是优化问题\ref{eq-primal}的对偶问题。
\begin{equation}
\label{eq-dual}
	\begin{split}
\max_{\bm \alpha} &\quad 2\bm \alpha^\mathrm T \mathbf{1}- \left\lVert
 \begin{matrix}
   \bm \alpha^\mathrm{T}\mathbf Y^\mathrm{T} \mathbf K_1 \mathbf Y  \bm \alpha \\
   \vdots \\
  \bm \alpha^\mathrm{T}\mathbf Y^\mathrm{T} \mathbf K_d \mathbf Y  \bm \alpha 
  \end{matrix}
  \right\rVert_p\\
  \text{s.t.} &\quad  \mathbf{0} \leq \bm \alpha  \leq \mathbf{C} 
  \end{split}
\end{equation}
其中, $p$和$q$满足共轭关系, 即$\frac{1}{p}+\frac{1}{q}=1$. 同时, $\mathbf Y = \text{diag}([y_1,\cdots,y_m])$, $\mathbf K_k$是由$\bm \Phi_k$定义的核函数(kernel).}
\item{ \textbf{[20pts]} 考虑在优化问题\ref{eq-dual}中, 当$p=1$时, 试化简该问题。}
\end{enumerate}

\begin{solution}
此处用于写解答(中英文均可)\\
(1)优化问题4.1的表达式中采用hinge损失。引入"松弛变量"$\epsilon_i \ge 0$,则优化问题4.1重写为:\\
\begin{equation}
\begin{split}
\min_{\mathbf w, \mu\in \Delta_q} &\quad \frac{1}{2} \sum_{k=1}^d \frac{1}{\mu_k}\lVert \mathbf w_k \rVert^2_2 + C\sum_{i=1}^m \epsilon_i\\
\text{s.t.} &\quad y_i(\sum_{k=1}^d  \mathbf w_k  \cdot \bm \Phi_k(\mathbf{x}_i)) \ge 1-\epsilon_i\\
&\quad \epsilon_i \ge 0\\
&\quad \mu_k \ge 0\\
&\quad \lVert \bm \mu \rVert = 1
\end{split}
\end{equation}
引入拉格朗日乘子$\alpha_i \ge 0, \beta_i \ge 0, \gamma_i \ge 0, h \ge 0$得:\\
$L = \frac{1}{2} \sum_{k=1}^d \frac{1}{\mu_k} \lVert \mathbf w_k \rVert^2_2 + C\sum_{i=1}^m \epsilon_i + \sum_{i=1}^m \alpha_i ( 1-\epsilon_i - y_i ( \sum_{k=1}^d \mathbf w_k \cdot \bm \Phi_k (\mathbf{x}_i) ) ) - \sum_{i=1}^m \beta_i \epsilon_i - \sum_{k=1}^d \gamma_k \mu_k + h(\lVert\bm\mu \rVert_q - 1)$\\
分别对$\mathbf w_k$, $\epsilon_i$ 和$\mu_k $求导,得:\\
$\frac{\partial L}{\partial \mathbf w_k} = 0 \Rightarrow \frac{\mathbf w_k}{\mu_k} = \sum_{i=1}^m \alpha_i y_i \bm\Phi_k(\mathbf{x}_i)$\\
$\frac{\partial L}{\partial \epsilon_i} = 0 \Rightarrow C = \alpha_i + \beta_i$\\
$\frac{\partial L}{\partial \mu_k} = 0 \Rightarrow \gamma_k = h \cdot \mu_k^{q-1} - \frac{\lVert \bm w_k \rVert^2_2}{2\mu_k^2}$\\
将上面的三个式子带入拉格朗日函数,得:\\
\begin{equation}
\begin{split}
L = \sum_{i=1}^m \alpha_i - \sum_{k=1}^d h \mu_k^q + h( (\sum_{k=1}^d \mu_k^q )^{1/q} - 1)\\
\end{split}
\end{equation}
$\because \lVert \bm \mu \rVert_q = 1$\\
$\therefore (\sum_{i=1}^d \mu_k^q)^{1/q} = 1$\\
$\therefore \sum_{i=1}^d \mu_k^q = 1$且$(\sum_{i=1}^d \mu_k^q)^{1/p} = 1$\\
代入上式,得:\\
$L = \sum_{i=1}^m \alpha_i - h$\\
$ = \sum_{i=1}^m \alpha_i - h \cdot (\sum_{i=1}^d \mu_k^q)^{1/p}$\\
$ = \sum_{i=1}^m \alpha_i - (\sum_{i=1}^d h^p\mu_k^q)^{1/p}$\\
$\because h \cdot \mu_k^{q-1} - \frac{\lVert \bm w_k \rVert^2_2}{2\mu_k^2} = \gamma_k $\\
$\therefore h \cdot \mu_k^q - \frac{\lVert \bm w_k \rVert^2_2}{2\mu_k} = \gamma_k \mu_k = 0$ ($\gamma_k \mu_k = 0$由KKT条件得)\\
$\therefore h \cdot \mu_k^{q-1} = \frac{\lVert \bm w_k \rVert^2_2}{2\mu_k^2}$\\
$\therefore h^p \cdot (\mu_k^{q-1})^p = (\frac{\lVert \bm w_k \rVert^2_2}{2\mu_k^2})^p$\\
$\because \frac{1}{p} + \frac{1}{q}  =1$\\
$\therefore q = p(q-1)$\\
$\therefore h^p \cdot \mu_k^q = (\frac{\lVert \bm w_k \rVert^2_2}{2\mu_k^2})^p$\\
代入L中即得:\\
$L =\sum_{i=1}^m \alpha_i - \frac{1}{2} \left(\sum_{k=1}^d  (\sum_{i=1}^m\sum_{j=1}^m \alpha_i\alpha_j y_i y_j \bm\Phi_k(\mathbf{x}_i) \bm\Phi_k(\mathbf{x}_j) )^p\right)^{1/p}$\\
上式整理可得原问题的对偶问题:\\
\begin{equation}
  \begin{split}
\max_{\bm \alpha} &\quad 2\sum_{i=1}^m \alpha_i - 1 \cdot \left(\sum_{k=1}^d \left( \sum_{i=1}^m\sum_{j=1}^m \alpha_i\alpha_j y_i y_j \bm\Phi_k(\mathbf{x}_i) \bm\Phi_k(\mathbf{x}_j)\right)^p\right)^{1/p}\\
  \text{s.t.} &\quad 0 \le \boldsymbol \alpha \le \mathbf{C}
  \end{split}
\end{equation}
将上式做一些形式上的化简,即得优化问题4.2:\\
\begin{equation}
  \begin{split}
\max_{\bm \alpha} &\quad 2\bm \alpha^\mathrm T \mathbf{1}- \left\lVert
 \begin{matrix}
   \bm \alpha^\mathrm{T}\mathbf Y^\mathrm{T} \mathbf K_1 \mathbf Y  \bm \alpha \\
   \vdots \\
  \bm \alpha^\mathrm{T}\mathbf Y^\mathrm{T} \mathbf K_d \mathbf Y  \bm \alpha 
  \end{matrix}
  \right\rVert_p\\
  \text{s.t.} &\quad  \mathbf{0} \leq \bm \alpha  \leq \mathbf{C} 
  \end{split}
\end{equation}
(2)当$p=1$,化简可得:\\
\begin{equation}
  \begin{split}
\max_{\bm \alpha} &\quad 2\bm \alpha^\mathrm T \mathbf{1}- \sum_{i=1}^d \sum_{i=1}^m\sum_{j=1}^m \alpha_i\alpha_j y_i y_j \bm\Phi_k(\mathbf{x}_i) \bm\Phi_k(\mathbf{x}_i)\\
  \text{s.t.} &\quad  \mathbf{0} \leq \bm \alpha  \leq \mathbf{C} 
  \end{split}
\end{equation}
\end{solution}
\end{document}