\documentclass[a4paper,UTF8]{article}
\usepackage{ctex}
\usepackage[margin=1.25in]{geometry}
\usepackage{color}
\usepackage{graphicx}
\usepackage{amssymb}
\usepackage{amsmath}
\usepackage{amsthm}
%\usepackage[thmmarks, amsmath, thref]{ntheorem}
\theoremstyle{definition}
\newtheorem*{solution}{Solution}
\newtheorem*{prove}{Proof}
\usepackage{multirow}

%--

%--
\begin{document}
\title{习题一}
\author{141220120, 徐世坚}
\maketitle


\section*{Problem 1}
若数据包含噪声,则假设空间中有可能不存在与所有训练样本都一致的假设,此时的版本空间是什么?在此情形下,试设计一种归纳偏好用于假设选择。
\begin{solution}
此时的版本空间可以定义为:能够正确判断样本最多的假设,或者是设定一个阈值,判断正确达到阈值及以上的假设。\\此时的归纳偏好可以定为:在一定正确率的前提下,采用奥卡姆剃刀原则,选择最简单的假设。
\end{solution}

\section*{Problem 2}
对于有限样例,请证明
\[
\text{AUC} = \frac{1}{m^+m^-}\sum_{x^+\in D^+}\sum_{x^-\in D^-}\left(\mathbb{I}(f(x^+)>f(x^-))+\frac{1}{2}\mathbb{I}(f(x^+)=f(x^-))\right)
\]

\begin{prove}
AUC=$\frac{1}{2}\sum_{i}^{m-1}(x_{i+1}-x_{i})(y_{i+1}+y_{i})$\\
=$\frac{1}{2}\sum_{(x_{i},y_{i})\in D^-}\frac{1}{m^-}(2y_{i})$\\
=$\frac{1}{m^-}\sum_{(x_{i},y_{i})\in D^-}y_{i}$\\
=$\frac{1}{m^-}\sum_{(x_{i},y_{i})\in D^-}\frac{k}{m^+}$\\
=$\frac{1}{m^-m^+}\sum_{x^-\in D^-}$(当前样本之前的真正例个数)\\
将样例按预测值从大到小排序,因为当前的样本是个反例,所以该样本之前的真正例的个数等于预测值比它大的正例的个数,对应于$\mathbb{I}(f(x^+)>f(x^-))$,当因为可能存在正例的预测值和当前的反例预测值相等,那么在排序中可能排在前也可能排在后,取平均就对应于$\frac{1}{2}\mathbb{I}(f(x^+)=f(x^-))$\\
$\therefore$上式=$\frac{1}{m^-m^+}\sum_{x^-\in D^-}\sum_{x^+\in D^+}(\mathbb{I}(f(x^+)>f(x^-))+\frac{1}{2}\mathbb{I}(f(x^+)=f(x^-)))$
\qed
\end{prove}

\section*{Problem 3} 
在某个西瓜分类任务的验证集中,共有10个示例,其中有3个类别标记为“1”,表示该示例是好瓜;有7个类别标记为“0”,表示该示例不是好瓜。由于学习方法能力有限,我们只能产生在验证集上精度(accuracy)为0.8的分类器。
\begin{itemize}
\item[(a)] 如果想要在验证集上得到最佳查准率(precision),该分类器应该作出何种预测?

此时的查全率(recall)和F1分别是多少?
\item[(b)] 如果想要在验证集上得到最佳查全率(recall),该分类器应该作出何种预测?

此时的查准率(precision)和F1分别是多少?
\end{itemize}
\begin{solution}
$\because$ accuracy=0.8\\
$\therefore$ 有8个样本分类正确\\
设$TP=x$, $x\in$ {0,1,2,3}; 设$TN=8-x$, $FN=3-x$, $FP=7-(8-x)=x-1$.\\
P=$\frac{TP}{TP+FP}=\frac{x}{2x-1}$,R=$\frac{TP}{TP+FN}=\frac{x}{3}$.
\item[(a)]当x=1,查准率最高,P=1\\
TP=1, FN=2, TP=0, TN=7\\
R=$\frac{1}{3}$\\
F1=$\frac{2\times P\times R}{P+R}$=0.5

\item[(b)]当x=3,查全率最高,R=1\\
TP=3, FN=0, FP=2, TN=5\\
p=$\frac{3}{5}$\\
F1=$\frac{2\times P\times R}{P+R}$=0.75
\end{solution}

\section*{Problem 4} 
在数据集$D_1,D_2,D_3,D_4,D_5$运行了$A,B,C,D,E$五种算法,算法比较序值表如表\ref{table:ranking}所示:
\begin{table}[h]
\centering
\caption{算法比较序值表} \vspace{2mm}\label{table:ranking}
\begin{tabular}{c|c c c c c}\hline
数据集 & 算法$A$ & 算法$B$  & 算法$C$  &算法$D$  &算法$E$ \\
\hline
$D_1$ & 2  & 3 &  1 &  5  & 4\\
$D_2$ & 5  & 4 &  2 &  3  & 1\\
$D_3$ & 4  & 5 &  1 &  2  & 3\\
$D_4$ & 2  & 3 &  1 &  5  & 4\\
$D_5$ & 3  & 4 &  1 &  5  & 2\\
\hline
平均序值 & 3.2 &  3.8 & 1.2 &  4 &  2.8 \\
\hline
\end{tabular}
\end{table}

使用Friedman检验$(\alpha=0.05)$判断这些算法是否性能都相同。若不相同,进行Nemenyi后续检验$(\alpha=0.05)$,并说明性能最好的算法与哪些算法有显著差别。
\begin{solution}
Friedman检验:$\tau_{\chi^2}=\frac{248}{25}$,$\tau_F=3.937$.它大于$\alpha=0.05$时的F检验临界值3.007,因此拒绝所有算法性能相同这个假设。\\
Nemenyi检验:$CD=2.728$, 算法C最好,它与算法D的差距超过临界值域,所以,最好的算法C和算法D有显著差别。

\end{solution}
\end{document}